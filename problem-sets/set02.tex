\documentclass{homework}
\course{Math 6701}
\author{Jim Fowler}
\usepackage{amsmath}
\DeclareMathOperator{\Mat}{Mat}
\DeclareMathOperator{\End}{End}
\DeclareMathOperator{\Hom}{Hom}
\DeclareMathOperator{\id}{id}
\DeclareMathOperator{\image}{im}
\DeclareMathOperator{\rank}{rank}
\DeclareMathOperator{\nullity}{nullity}
\DeclareMathOperator{\trace}{tr}
\DeclareMathOperator{\Spec}{Spec}
\DeclareMathOperator{\Sym}{Sym}
\DeclareMathOperator{\pf}{pf}
\DeclareMathOperator{\Ortho}{O}
\DeclareMathOperator{\diam}{diam}
\DeclareMathOperator{\Gr}{Gr}
\DeclareMathOperator{\SO}{SO}
\DeclareMathOperator{\Real}{Re}
\DeclareMathOperator{\Imag}{Im}
\DeclareMathOperator{\dR}{dR}

\newcommand{\Proj}{\mathbb{P}}
\newcommand{\RP}{\mathbb{R}P}
\newcommand{\CP}{\mathbb{C}P}

\DeclareMathOperator{\Arg}{Arg}

\newcommand{\C}{\mathbb{C}}

\DeclareMathOperator{\sla}{\mathfrak{sl}}
\newcommand{\norm}[1]{\left\lVert#1\right\rVert}
\newcommand{\transpose}{\intercal}

\newcommand{\conj}[1]{\bar{#1}}
\newcommand{\abs}[1]{\left|#1\right|}

\begin{document}
\maketitle

\begin{inspiration}
  A plan is just a tangent vector on the manifold of reality.
  \byline{``Scratch'' Garrison} % who is this?!
\end{inspiration}

\section{Terminology}

\begin{problem}
  Define $T_p M$, the \textbf{tangent space} to a manifold $M$ at the
  point $p$.
\end{problem}

\begin{problem}
  Define \textbf{submersion} and \textbf{immersion} and \textbf{local diffeomorphism}.
\end{problem}

\section{Numericals}

\begin{problem}
  Consider the circle $S^1 \ni x_1, x_2$ with $x_1 \neq x_2$, giving
  rise to an open cover $\mathcal{U} = \{ U_1, U_2 \}$ with
  $U_i = S^1 \setminus \{x_i\}$ and charts $(U_i,\varphi_i)$.

  Explicitly describe a partition of unity subordinate to
  $\mathcal{U}$ in the sense of describing smooth functions
  $f_i : S^1 \to \R$ with the support of $f_i$ contained in $U_i$.
\end{problem}

\begin{problem}
  The Lie group $O(n)$ consists of orthogonal $n \times n$ matrices.

  Describe $T_{\id} O(n)$ as a collection of certain matrices.
\end{problem}

\begin{problem}
  Let $H : S^3 \to S^2$ be the Hopf map.

  Compute $dH_p : T_p S^3 \to T_{H(p)} S^2$ and verify that $H$ is a
  submersion.
\end{problem}

\section{Exploration}

\begin{problem}
  The textbook explored the local structure of immersions; let's do submersions.
  
  The smooth map $f : X \to Y$ is a \textbf{local submersion} at
  $x \in X$ if $df_x : T_x X \to T_{f(x)} Y$ is surjective.  Show that
  there are charts $(U,\phi)$ with $x \in U \subset X$ and $(V,\psi)$
  with $f(x) \in V \subset Y$ so that
  $\psi \circ f \circ \phi^{-1} : \phi(U) \to \psi(V)$ is the
  canonical projection map
  $(v_1,\ldots,v_{\dim X}) \mapsto (v_1,\ldots,v_{\dim Y})$.
\end{problem}

\begin{problem}
  For a smooth map $f : X \to Y$, we say that $y \in Y$ is a
  \textbf{regular value} if, whenever $x \in f^{-1}(y)$, the map
  $df_x : T_x X \to T_y Y$ is surjective.  When $y \in Y$ is not a
  regular value, we say it is a \textbf{critical value}.

  Show that $f^{-1}(y)$ is a submanifold when $y$ is a regular value.

  What is the dimension of this submanifold?
\end{problem}


\section{Prove or Disprove and Salvage if Possible (PODASIP)}

Recall that we used bump functions (among other things) to show that
every compact $n$-manifold embeds in $S^N$ for some large $N$.

\begin{problem}
  Every $n$-manifold smoothly surjects onto $S^n$.
\end{problem}

\begin{problem}
  If $f : V \to W$ is a linear map and $v \in V$, then $df_v$ is $f$.

  (In other words, the derivative of a linear map is itself.)
\end{problem}

\begin{problem}
  The intersection of two submanifolds is a submanifold.
\end{problem}

\end{document}
