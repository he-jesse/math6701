\documentclass{homework}
\course{Math 6701}
\author{Jim Fowler}
\usepackage{amsmath}
\DeclareMathOperator{\Mat}{Mat}
\DeclareMathOperator{\End}{End}
\DeclareMathOperator{\Hom}{Hom}
\DeclareMathOperator{\id}{id}
\DeclareMathOperator{\image}{im}
\DeclareMathOperator{\rank}{rank}
\DeclareMathOperator{\nullity}{nullity}
\DeclareMathOperator{\trace}{tr}
\DeclareMathOperator{\Spec}{Spec}
\DeclareMathOperator{\Sym}{Sym}
\DeclareMathOperator{\pf}{pf}
\DeclareMathOperator{\Ortho}{O}
\DeclareMathOperator{\diam}{diam}
\DeclareMathOperator{\Gr}{Gr}
\DeclareMathOperator{\SO}{SO}
\DeclareMathOperator{\Real}{Re}
\DeclareMathOperator{\Imag}{Im}
\DeclareMathOperator{\dR}{dR}

\newcommand{\Proj}{\mathbb{P}}
\newcommand{\RP}{\mathbb{R}P}
\newcommand{\CP}{\mathbb{C}P}

\DeclareMathOperator{\Arg}{Arg}

\newcommand{\C}{\mathbb{C}}

\DeclareMathOperator{\sla}{\mathfrak{sl}}
\newcommand{\norm}[1]{\left\lVert#1\right\rVert}
\newcommand{\transpose}{\intercal}

\newcommand{\conj}[1]{\bar{#1}}
\newcommand{\abs}[1]{\left|#1\right|}

\begin{document}
\maketitle

\begin{inspiration}
  Now that we have seen that superluminal communication is not
  possible, we turn to the more prosaic task of writing tensor
  products using standard bases.  \byline{Chris Bernhardt, in
    \textit{Quantum Computing for Everyone}, page 64.}
\end{inspiration}

\section{Terminology}

\begin{problem}
  What is the \textbf{bracket} of vector fields?
\end{problem}

\begin{problem}
  What is meant by the \textbf{codimension} of a submanifold $N^n \subset M^m$?
\end{problem}

\begin{problem}
  What does it mean to say that a smooth map $f : X \to Y$ is
  \textbf{transverse} to a submanifold $Z \subset Y$?
\end{problem}

\section{Numericals}

\begin{problem}
  For a nonzero vector $v \in \R^3$, let $R(v,\theta)$ denote
  counterclockwise rotation around the axis $v$.  Define a map $f :
  S^1 \times S^1 \times S^1 \to \SO(3)$ by sending
  $(\theta_1,\theta_2,\theta_3) \in (S^1)^3$ to the rotation
  $$
  R(\textbf{x},\theta_1) \circ R(\textbf{y},\theta_2) \circ R(\textbf{x},\theta_3) \in \SO(3).
  $$
  Here, $\textbf{x}$ and $\textbf{y}$ denote the unit vectors
  $(1,0,0)$ and $(0,1,0)$, respectively.  Show that $f$ is not
  a submersion.

  \textit{Note:} When you are controlling a system, you may have only
  infinitesimal control over the inputs---and if the function is not a
  submersion, you then do not have complete (albeit, infinitesimal!)
  control over the outputs.  This problem---known as gimbal lock---was
  experienced by the astronauts of Apollo~11.
\end{problem}

\section{Exploration}

\begin{problem}
  Consider $M = S^1 \times \R^2$; a point $(\theta,x) \in M$
  corresponds to a position of an automobile $x \in \R^2$, and the
  angle $\theta$ of its wheels.

  Suppose $F \in \Gamma(TM)$ is the ``drive in the direction of the
  wheels'' vector field, and $T \in \Gamma(TM)$ is the ``turn steering
  wheel'' vector field.  What is $[F,T]$ and how does this help with
  parking the car?
\end{problem}

\begin{problem}
  Let $V = \R^3$, and consider the map $s : V \to \Sym^2
  V$ which sends $s(v)$ to $v \cdot v$.  This map $s$ is not linear, but
  $s$ induces a map $\Proj(s) : \Proj(V) \to \Proj(\Sym^2 V)$ sending lines to lines.

  Is $\Proj(s)$ a smooth map?  Compute its derivative
  $\Proj(s)_\star : T\RP^2 \to T\RP^5$.

  Is $\Proj(s)$ a smooth embedding?
\end{problem}

\section{Prove or Disprove and Salvage if Possible (PODASIP)}

\begin{problem}
  The real projective space $\RP^n$ is orientable.
\end{problem}

\begin{problem}
  There is a smooth vector field on $S^2$ which vanishes at exactly one point.
\end{problem}


\end{document}
