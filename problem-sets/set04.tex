\documentclass{homework}
\course{Math 6701}
\author{Jim Fowler}
\usepackage{amsmath}
\DeclareMathOperator{\Mat}{Mat}
\DeclareMathOperator{\End}{End}
\DeclareMathOperator{\Hom}{Hom}
\DeclareMathOperator{\id}{id}
\DeclareMathOperator{\image}{im}
\DeclareMathOperator{\rank}{rank}
\DeclareMathOperator{\nullity}{nullity}
\DeclareMathOperator{\trace}{tr}
\DeclareMathOperator{\Spec}{Spec}
\DeclareMathOperator{\Sym}{Sym}
\DeclareMathOperator{\pf}{pf}
\DeclareMathOperator{\Ortho}{O}
\DeclareMathOperator{\diam}{diam}
\DeclareMathOperator{\Gr}{Gr}
\DeclareMathOperator{\SO}{SO}
\DeclareMathOperator{\Real}{Re}
\DeclareMathOperator{\Imag}{Im}
\DeclareMathOperator{\dR}{dR}

\newcommand{\Proj}{\mathbb{P}}
\newcommand{\RP}{\mathbb{R}P}
\newcommand{\CP}{\mathbb{C}P}

\DeclareMathOperator{\Arg}{Arg}

\newcommand{\C}{\mathbb{C}}

\DeclareMathOperator{\sla}{\mathfrak{sl}}
\newcommand{\norm}[1]{\left\lVert#1\right\rVert}
\newcommand{\transpose}{\intercal}

\newcommand{\conj}[1]{\bar{#1}}
\newcommand{\abs}[1]{\left|#1\right|}

\begin{document}
\maketitle

\begin{inspiration}
  % https://twitter.com/metaweta
  Old MacDonald had a form, $e_i \wedge e_i = 0$. \byline{Mike Stay}
\end{inspiration}

\section{Terminology}

\begin{problem}
  What is an \textbf{exact} differential form?  A \textbf{closed} form?
\end{problem}

\section{Numericals}

% Baker-Campbell-Hausdorff for the Heisenberg group
\begin{problem} Let $H$ be the Lie group of $3 \times 3$ matrices of
the form 
  $\displaystyle\begin{pmatrix}
    1 & a & b \\
    0 & 1 & c \\
    0 & 0 & 1
  \end{pmatrix}$ for $a, b, c \in \R$, and set $\mathfrak{h} := T_e H$
so that $v \in \mathfrak{h}$ gives rise to the left-invariant vector
field $X_v$ with $X_v(e) = v$; the function $\theta_{v} : H \times \R
\to H$ is the flow along this vector field, and the exponential map
$\exp : \mathfrak{h} \to H$ is defined by $\exp(v) = \theta_v(e,1)$.

When $\exp c = (\exp a) \cdot (\exp b)$, find a formula for $c$ in
terms of $a, b \in \mathfrak{h}$.  Your formula will permit you to do
calculations in the Heisenberg group $H$ by doing calculations in its Lie
algebra $\mathfrak{h}$.
\end{problem}

\section{Exploration}

\begin{problem}
Write ``$H^k(M) = 0$'' to mean that every closed $k$-form on $M$ is exact.

  Suppose for open subsets $U, V \subset M$ with $U \cap V =
  \varnothing$, we have $H^1(U) = 0$ and $H^1(V) = 0$.

  What additional condition might you impose to ensure $H^1(U \cup V)
\neq 0$?
\end{problem}


\section{Prove or Disprove and Salvage if Possible (PODASIP)}

\begin{problem} Fix pointwise linearly independent forms $\omega_1
\ldots, \omega_n \in \Omega^1(M)$.

  Then $\eta_1, \ldots, \eta_n \in \Omega^1(M)$ satisfy $\sum_i \eta_i \wedge \omega_i = 0$ iff there exist $f_{ij} \in C^\infty(M)$ so that $\eta_i = \sum_j f_{ij} \omega_j$ and $f_{ij} = f_{ji}$.
\end{problem}

\begin{problem} For vector fields $X, Y \in \Gamma(TM)$ and a 1-form
$\omega \in \Omega^1(M)$,
\[
  d\omega(X,Y) = X(\omega (Y)) - Y(\omega(X)) - \omega([X,Y]).
\]
\end{problem}

% Credit to https://twitter.com/JulesJacobs5/status/1437415867185344517
\begin{problem} Suppose $U \subset \R^2$ and $x,y,z: U \to \R$ are
smooth functions with the property that any pair of the two provides
local coordinates on $U$.  Then
  \[
    \left. \frac{\partial x}{\partial y} \right|_{z} 
    \left. \frac{\partial y}{\partial z} \right|_{x} 
    \left. \frac{\partial z}{\partial x} \right|_{y} = 1. % salvage to -1 
  \] \textit{Hint:} The notation $\left. \frac{\partial x}{\partial y}
\right|_{z}$ means the partial derivative of $x$ with respect to $y$,
when $x$ is expressed as a function of $y$ and $z$.  Use the fact that
$\left. \frac{\partial x}{\partial y} \right|_{z} = \frac{dx \wedge
dz}{dy \wedge dz}$.
\end{problem}

\newcommand{\respectively}[2]{$\genfrac\{\}{0pt}{0}{\mbox{#1}}{\mbox{#2}}$}

\begin{problem} If $\omega \in \Omega^k(M)$ and $\eta \in
\Omega^{\ell}(M)$ are \respectively{closed}{exact}, then $\omega
\wedge \eta$ is \respectively{closed}{exact}.
\end{problem}


\end{document}
