\documentclass{homework}
\course{Math 6701}
\author{Jim Fowler}
\usepackage{amsmath}
\DeclareMathOperator{\Mat}{Mat}
\DeclareMathOperator{\End}{End}
\DeclareMathOperator{\Hom}{Hom}
\DeclareMathOperator{\id}{id}
\DeclareMathOperator{\image}{im}
\DeclareMathOperator{\rank}{rank}
\DeclareMathOperator{\nullity}{nullity}
\DeclareMathOperator{\trace}{tr}
\DeclareMathOperator{\Spec}{Spec}
\DeclareMathOperator{\Sym}{Sym}
\DeclareMathOperator{\pf}{pf}
\DeclareMathOperator{\Ortho}{O}
\DeclareMathOperator{\diam}{diam}
\DeclareMathOperator{\Gr}{Gr}
\DeclareMathOperator{\SO}{SO}
\DeclareMathOperator{\Real}{Re}
\DeclareMathOperator{\Imag}{Im}
\DeclareMathOperator{\dR}{dR}

\newcommand{\Proj}{\mathbb{P}}
\newcommand{\RP}{\mathbb{R}P}
\newcommand{\CP}{\mathbb{C}P}

\DeclareMathOperator{\Arg}{Arg}

\newcommand{\C}{\mathbb{C}}

\DeclareMathOperator{\sla}{\mathfrak{sl}}
\newcommand{\norm}[1]{\left\lVert#1\right\rVert}
\newcommand{\transpose}{\intercal}

\newcommand{\conj}[1]{\bar{#1}}
\newcommand{\abs}[1]{\left|#1\right|}

\begin{document}
\maketitle

\begin{inspiration} As I see it at last it was my lot to plant the
harpoon of algebraic topology into the body of the whale of algebraic
geometry. But I must not push the metaphor too far.  \byline{Solomon
Lefschetz, in \textit{A page of mathematical autobiography}, 1968}
%Bull. Amer. Math. Soc. 74 (1968), 854-879.}
\end{inspiration}

\textbf{Scheduling note:} Thursday and Friday are holidays. \\[1ex] This problem set can be turned in on Wednesday or on Monday.

\section{Terminology}

\begin{problem} Let $M, N$ be connected, oriented, closed smooth
$n$-manifolds. \\ For a smooth map $f : M \to N$, what is the
\textbf{degree} of the map $f$?
\end{problem}

\begin{problem}
 Define $H_c^k(M)$, de Rham cohomology with compact support. 
\end{problem}

\begin{problem}
  Again let $M, N$ be connected, oriented, closed smooth $n$-manifolds.\\  What is the \textbf{connected sum} of $M$ and $N$, written $M \# N$?
\end{problem}

\section{Numericals}

\begin{problem}
Compute $H_{\dR}^k(\Sigma_g)$.  Recall that $\Sigma_g = T^2 \# \cdots \# T^2$ with $g$ copies of the torus $T^2$.
\end{problem}

\section{Exploration}

\begin{problem} For a closed, connected, oriented manifold $M$, our
solution to \ref{degree-one-to-sphere} produced a degree one map $M
\to S^n$. \\[1ex]  Is there necessarily a degree one map $S^n \to M$?
\end{problem}

\begin{problem} In \ref{no-retraction}, we saw that there is no
retraction of a disk onto its boundary.  \\[1ex] For a connected,
oriented, compact, smooth manifold $M$ with boundary $\partial M$, is
there a retraction of $M$ onto $\partial M$?
\end{problem}

\section{Prove or Disprove and Salvage if Possible (PODASIP)}

\begin{problem}
  Let $X$ be the disjoint union of countably many points.
  Then
 \[H^0(X \times X) \cong H^0(X) \otimes H^0(X).\]
This is Example~9.14 from Bott--Tu.
\end{problem}

\begin{problem} % salvage to connected
If the smooth $n$-manifold $M$ is not orientable, then $H_{\dR}^n(M) = 0$.
\end{problem}

\begin{problem} For a map $f : M \to N$, there are induced maps
$f^\star : H_c^k(N) \to H_c^k(M)$ and $f_\star : H_c^k(M) \to
H_c^k(N)$.
\end{problem} % proper maps pullback, covariant for inclusions


\end{document}
