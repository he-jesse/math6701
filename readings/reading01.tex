\documentclass{homework}
\usepackage{amsmath}
\DeclareMathOperator{\Mat}{Mat}
\DeclareMathOperator{\End}{End}
\DeclareMathOperator{\Hom}{Hom}
\DeclareMathOperator{\id}{id}
\DeclareMathOperator{\image}{im}
\DeclareMathOperator{\rank}{rank}
\DeclareMathOperator{\nullity}{nullity}
\DeclareMathOperator{\trace}{tr}
\DeclareMathOperator{\Spec}{Spec}
\DeclareMathOperator{\Sym}{Sym}
\DeclareMathOperator{\pf}{pf}
\DeclareMathOperator{\Ortho}{O}
\DeclareMathOperator{\diam}{diam}
\DeclareMathOperator{\Gr}{Gr}
\DeclareMathOperator{\SO}{SO}
\DeclareMathOperator{\Real}{Re}
\DeclareMathOperator{\Imag}{Im}
\DeclareMathOperator{\dR}{dR}

\newcommand{\Proj}{\mathbb{P}}
\newcommand{\RP}{\mathbb{R}P}
\newcommand{\CP}{\mathbb{C}P}

\DeclareMathOperator{\Arg}{Arg}

\newcommand{\C}{\mathbb{C}}

\DeclareMathOperator{\sla}{\mathfrak{sl}}
\newcommand{\norm}[1]{\left\lVert#1\right\rVert}
\newcommand{\transpose}{\intercal}

\newcommand{\conj}[1]{\bar{#1}}
\newcommand{\abs}[1]{\left|#1\right|}
\author{Jim Fowler}
\course{Math 6701}
\date{Week 1: Manifolds}

\begin{document}
\maketitle

For this first week, our goal is to become acquainted with the primary
object of our study: manifolds, and in particular, differentiable
manifolds.  These objects are everywhere in mathematics; given their
ubiquity, some of you will have already encountered manifolds.  For
those who are already familiar with these objects, your goal will be
dig deeper.

From Morita's \textit{Geometry of Differential Forms}, read
\begin{itemize}
\item 1.1 What is a manifold?
\item 1.2 Definition and examples of manifolds
\end{itemize}
Or from Lee's \textit{Introduction to Smooth Manifolds}, read
\begin{itemize}
\item 1 Smooth Manifold
\end{itemize}

When you dig into a new book or a new subject, it can be helpful to
have some vision of the overall structure of your upcoming journey.
As is usual in mathematics, we'll be exploring our objects from an
analytic, an algebraic, and a geometric perspective.  In this case,
the analytic and geometric perspectives are clearer: differentiable
manifolds provide enough structure to do calculus, and when we tackle
topics like curvature, we are undeniably doing geometry.

So we begin our journey with manifolds as inherently geometric
objects, and then study ``differential forms'' to understand how some
analysis shines light on the geometry.  It's at that point that we
will meet differential forms and then cohomology, that is, we'll be
seeing the first glimpses of \textit{algebraic} topology.  Cohomology
provides a receptable for invariants, so-called characteristic
classes, associated with vector bundles and with fiber bundles.

\end{document}
