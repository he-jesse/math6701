\documentclass{homework}
\usepackage{amsmath}
\DeclareMathOperator{\Mat}{Mat}
\DeclareMathOperator{\End}{End}
\DeclareMathOperator{\Hom}{Hom}
\DeclareMathOperator{\id}{id}
\DeclareMathOperator{\image}{im}
\DeclareMathOperator{\rank}{rank}
\DeclareMathOperator{\nullity}{nullity}
\DeclareMathOperator{\trace}{tr}
\DeclareMathOperator{\Spec}{Spec}
\DeclareMathOperator{\Sym}{Sym}
\DeclareMathOperator{\pf}{pf}
\DeclareMathOperator{\Ortho}{O}
\DeclareMathOperator{\diam}{diam}
\DeclareMathOperator{\Gr}{Gr}
\DeclareMathOperator{\SO}{SO}
\DeclareMathOperator{\Real}{Re}
\DeclareMathOperator{\Imag}{Im}
\DeclareMathOperator{\dR}{dR}

\newcommand{\Proj}{\mathbb{P}}
\newcommand{\RP}{\mathbb{R}P}
\newcommand{\CP}{\mathbb{C}P}

\DeclareMathOperator{\Arg}{Arg}

\newcommand{\C}{\mathbb{C}}

\DeclareMathOperator{\sla}{\mathfrak{sl}}
\newcommand{\norm}[1]{\left\lVert#1\right\rVert}
\newcommand{\transpose}{\intercal}

\newcommand{\conj}[1]{\bar{#1}}
\newcommand{\abs}[1]{\left|#1\right|}
\author{Jim Fowler}
\course{Math 6701}
\date{Week 2: Tangent vectors}

\begin{document}
\maketitle

Last week, we saw the precise definition for \textbf{smooth manifold}
in terms of charts and atlases.  This week, we continue translating
our ``advanced calculus'' ideas about $\mathbb{R}^n$ to ideas about
$n$-manifolds.  For instance, we are familiar with tangent vectors;
this week, we learn how to make sense of tangent vectors to a smooth
manifold, even when that manifold does not come embedded in some
ambient space---although we will end the week by showing that our
abstractly defined smooth manifolds \textit{do} embed in some
$\mathbb{R}^n$.

To get started read
\begin{itemize}
\item 1.3 Tangent vectors and tangent spaces
\end{itemize}
from Morita's \textit{Geometry of Differential Forms}.  Armed with a
notion of tangent vector for smooth manifold, we will be able to
define $df$ for a function $f : M \to N$ between smooth manifolds,
which then will permit us to define \textbf{immersions} and
\textbf{submersions}, key notions of differential topology.

And although this week is called ``tangent vectors'' it is really
about much more.  We'll see new examples like real (and complex!)
\textbf{projective space}.  Notions like \textbf{submanifold} will
require careful definitions.  Richer structures on manifolds will make
their first appearance, like \textbf{Lie groups} and \textbf{complex
  manifolds}.  With all this, it still may feel like the ``desert of
definitions'' but important constructions like \textbf{partition of
  unity} will be a brief oasis in that desert.

\end{document}
