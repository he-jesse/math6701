\documentclass{homework}
\usepackage{amsmath}
\DeclareMathOperator{\Mat}{Mat}
\DeclareMathOperator{\End}{End}
\DeclareMathOperator{\Hom}{Hom}
\DeclareMathOperator{\id}{id}
\DeclareMathOperator{\image}{im}
\DeclareMathOperator{\rank}{rank}
\DeclareMathOperator{\nullity}{nullity}
\DeclareMathOperator{\trace}{tr}
\DeclareMathOperator{\Spec}{Spec}
\DeclareMathOperator{\Sym}{Sym}
\DeclareMathOperator{\pf}{pf}
\DeclareMathOperator{\Ortho}{O}
\DeclareMathOperator{\diam}{diam}
\DeclareMathOperator{\Gr}{Gr}
\DeclareMathOperator{\SO}{SO}
\DeclareMathOperator{\Real}{Re}
\DeclareMathOperator{\Imag}{Im}
\DeclareMathOperator{\dR}{dR}

\newcommand{\Proj}{\mathbb{P}}
\newcommand{\RP}{\mathbb{R}P}
\newcommand{\CP}{\mathbb{C}P}

\DeclareMathOperator{\Arg}{Arg}

\newcommand{\C}{\mathbb{C}}

\DeclareMathOperator{\sla}{\mathfrak{sl}}
\newcommand{\norm}[1]{\left\lVert#1\right\rVert}
\newcommand{\transpose}{\intercal}

\newcommand{\conj}[1]{\bar{#1}}
\newcommand{\abs}[1]{\left|#1\right|}
\author{Jim Fowler}
\course{Math 6701}
\date{Week 3: Vector fields}

\begin{document}
\maketitle

At the end of last week, we saw tangent vectors and recognized how we
could use those tangent vectors to do calculus, compute derivatives,
and ultimately define ``embeddings'' and verify that our
abstractly-built manifolds via charts give rise to the same objects as
embedded manifolds.

This week, we assemble tangent vectors together into \textbf{vector
  fields}, which are \textit{sections of the tangent bundle} meaning
that those tangent spaces $T_p M$ also can be assembled together into
the tangent bundle $TM$.  Think back to vector fields in a
differential equations course, and recall how useful it was to
consider both a point and a vector at the same time, i.e., for setting
up initial conditions.

From Morita's \textit{Geometry of Differential Forms}, read
\begin{itemize}
\item 1.4 Vector fields
\end{itemize}
Vector fields provide some fun examples; the homework will encourage
you to think about vector fields on $S^2$.  And a with a notion of
vector fields, we can also discuss \textbf{orientation} to determine
when a local notion of handedness can be extended over a manifold.
And if tangent bundles let us do linear algebra over a manifold, then
thinking about orientations suggests doing multilinear algebra over a
manifold---exactly why next week's theme is exterior algebra.

\end{document}
