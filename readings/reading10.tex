\documentclass{homework}
\usepackage{amsmath}
\DeclareMathOperator{\Mat}{Mat}
\DeclareMathOperator{\End}{End}
\DeclareMathOperator{\Hom}{Hom}
\DeclareMathOperator{\id}{id}
\DeclareMathOperator{\image}{im}
\DeclareMathOperator{\rank}{rank}
\DeclareMathOperator{\nullity}{nullity}
\DeclareMathOperator{\trace}{tr}
\DeclareMathOperator{\Spec}{Spec}
\DeclareMathOperator{\Sym}{Sym}
\DeclareMathOperator{\pf}{pf}
\DeclareMathOperator{\Ortho}{O}
\DeclareMathOperator{\diam}{diam}
\DeclareMathOperator{\Gr}{Gr}
\DeclareMathOperator{\SO}{SO}
\DeclareMathOperator{\Real}{Re}
\DeclareMathOperator{\Imag}{Im}
\DeclareMathOperator{\dR}{dR}

\newcommand{\Proj}{\mathbb{P}}
\newcommand{\RP}{\mathbb{R}P}
\newcommand{\CP}{\mathbb{C}P}

\DeclareMathOperator{\Arg}{Arg}

\newcommand{\C}{\mathbb{C}}

\DeclareMathOperator{\sla}{\mathfrak{sl}}
\newcommand{\norm}[1]{\left\lVert#1\right\rVert}
\newcommand{\transpose}{\intercal}

\newcommand{\conj}[1]{\bar{#1}}
\newcommand{\abs}[1]{\left|#1\right|}
\author{Jim Fowler}
\course{Math 6701}
\date{Week 10: Harmonic forms}

\usepackage{draftwatermark}
\SetWatermarkText{Draft}
\SetWatermarkScale{5}

\begin{document}
\maketitle

This marks a departure from what we've been studying!  With some
cohomological methods in our toolkit, we now consider \textit{harmonic
forms}. What are these?

But first, what's the problem that harmonic forms are trying to solve?
A cohomology class is first of all just that---a equivalence class.
We saw when showing that the Hopf invariant is well-defined the
bookkeeping required to handle the fact that the \textit{same} class
might be represented by different forms.  If only there were a
canonical repersentative in each class!

In the presence of a Riemannian metric, we can trade covectors for
vectors.  The resulting relationship between 1-forms and vector fields
permits us to solve the aforementioned problem: we can choose a
canonical representative by choosing a form which is
\textit{harmonic}, in the sense that the Laplacian $\Delta$ of the
form is zero.

To dig into this, open Morita's \textit{Geometry of Differential Forms} and read
\begin{itemize}
\item 4.1 Differential forms on Riemannian manifolds
\item 4.2 Laplacian and harmonic forms
\item 4.3 The Hodge theorem
\end{itemize} As applications, we'll again discuss Poincar\'e duality
and the K\"unneth formula and Euler characteristic, and finally meet
the \textbf{signature}, an interesting invariant.




\end{document}
